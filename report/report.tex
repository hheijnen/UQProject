\documentclass[11pt, a4paper]{article}

\usepackage{color}

\usepackage{amsmath}
\usepackage{amssymb}
\usepackage[pdftex]{graphicx}
\usepackage{alltt}
\usepackage{caption}
\usepackage{subcaption}
\usepackage{epstopdf}


\author{...}
\title{Epidemic models of Flu outbreaks}
\begin{document}
\maketitle

\section{Model description}
\subsection{The SIR model}
\paragraph{}
The Susceptible-Infected-Recovered (SIR) model is a deterministic model that can be used to describe the influenza transmission. We consider the version proposed by Coehlo et al. [2].

\begin{equation}
\frac{dS}{dt} = - \lambda S
\end{equation}
\begin{equation}
\frac{dI}{dt} = \lambda S - \tau R
\end{equation}
\begin{equation}
\frac{dR}{dt} = \tau R
\end{equation}

with \[ \lambda = \beta (\alpha I + m) \]

where $S$ is the normalized number of susceptible individuals, $I$is the normalized number of infected individuals, $R$ is the normalized number of recovered individuals, $\tau $ is the recovery rate, $\beta $ is the transmission rate, $\alpha$ is the ratio of symptomatic infection and m is the infectious migration rate.

The expression of the effective reproduction number (at the beginning of each season) is 
\begin{equation}
R_e = \frac{\beta \alpha}{\tau} S_o
\end{equation}

where $S_o$ is the number of the susceptible individuals at the beginning of each new period.

The set of parameters to be determined in this case is : $ { \alpha, m, R_e, S_o, \tau }$. The following parameters $m, R_e, \tau$ do not depend on the season considered, whereas the parameters $\alpha$ and $S_o$ change at each season.

\subsection{The SEIR model}
\paragraph{}
The Susceptible-Exposed-Infected-Recovered (SEIR) model is an extension of the SIR model. We added the Exposed category to the previous system.

\begin{equation}
\frac{dS}{dt} = - \lambda S
\end{equation}
\begin{equation}
\frac{dE}{dt} = \lambda S - \gamma E
\end{equation}
\begin{equation}
\frac{dI}{dt} = \gamma E - \tau R
\end{equation}
\begin{equation}
\frac{dR}{dt} = \tau R
\end{equation}

with \[ \lambda = \beta (\alpha I + m) \]

where $E$ is the normalized number of exposed individuals and $\gamma$ is the transition rate from latent to infected state.

The set of parameters to be determined in this model is : $ { \alpha, \gamma, m, R_e, S_o, \tau }$. Here, we suppose that $\gamma$ is constant for all the seasons.

\section{Likelihood}
\paragraph{}
The likelihood of the parameters is 
\begin{equation}
L(\Theta) =  p(D|\Theta, I) = \prod_{k=1}^N p(y_k | \Theta,I) = 
\end{equation}

assuming gaussian errors
\begin{equation}
p(y_k | \Theta,I) = \frac{1}{\sqrt{2\pi}\sigma} \exp{-\frac{1}{2\sigma^2}[y_k - F(t_k)]}
\end{equation}

where $F(x_k) $is the number of infected people at time $t_k$ predicted by the model and $y_k$ is the observed data.

As

\section{Priors}
\paragraph{}
Thanks to the article by Coehlo et al. [2], it is possible to use informative priors for the SIR model. However, the $\gamma$ parameter has an uninformative prior (a uniform distribution between 0 and 1). Moreover, the prior over \tau is considered 

Here are the priors used for each parameter : 
\begin{itemize}
\item $\alpha \sim \mathcal{U}(0, 0.4)$ 
\item $\gamma \sim \mathcal{U}(0, 1)$
\item $m \sim  \mathcal{U}(0, 4e-6)$
\item $R_e \sim \mathcal{U}(1, 1.4)$
\item $ \tau \sim \mathcal{U}(1, 2)$
\end{itemize}



\end{document}
